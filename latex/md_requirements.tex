\subsubsection*{Terminology}

The following terms are used in this specification\+:
\begin{DoxyItemize}
\item The term “autonomous”, in this case, means that no commands can be transmitted to your robot from any outside agency (especially from a human or computer or other controller) and all sensors used in the contest must be physically attached to your robot. No wired connections are allowed between any outside agency and your robot.
\item The term “course” refers to the area in which the contest takes place.
\item The term “tape line” refers to an oval of white tape that runs from a start point around the oval, back to the start point (which is now the finish point). All targets will be placed outside of the oval.
\item The term “target” refers to the object you are required to pick up and dispose of (syringe or, alternatively, a pen or pencil).
\item The term “decoy” refers an object on the course that is not a target. A decoy will be less than 2 cm tall.
\item The term “obstacle” refers to an object on the course that your robot must avoid running into. An obstacle will be at least 15 cm tall. A typical obstacle would be a cardboard box.
\item The term “finish the course” will mean that your robot traverses the oval at least once. Note\+: Your robot will have to leave the tape line to pick up targets, but it should eventually either find another target or return to the tape line. The tape line is your navigation aid.
\item The term “contact a target” will mean to touch a target with your pick-\/up mechanism in such a way as to move it. Note\+: moving a target with a robot wheel or track does not count as a contact.
\item The term “participate” will mean that you either finish the course or contact a target.
\item The term “acquire a target” means your robot has reported to its data logger that it has identified a target and reports an accurate position for that target. The term “acquire a decoy” means your robot has reported to its data logger that it has acquired a target that turns out to be a decoy.
\item The term “pick up a target” refers to your robot picking up a target off the course surface.
\item The term “dispose of a target” refers to your robot placing the target in container on your robot.
\item A robot is “stationary” if its wheels are not rotating and its arm is not rotating about its vertical axis.
\end{DoxyItemize}

\subsubsection*{Rules of the Game}


\begin{DoxyItemize}
\item You will be given two test runs, one per day over two class periods. The dates will be firmly established by midterm time.
\item All tests will be conducted indoors.
\item A somewhat different course may be laid out each day. The layout will consist of\+:
\begin{DoxyItemize}
\item A tape line; this will serve as your navigation maker. Since we will be indoors, we won’t have G\+PS; the tape line will serve as your navigation reference.
\item A number of targets will be placed within 1 meter of the tape line; you will have to leave the tape line to pick up your targets.
\item A number of decoys will be placed within 1 meter of the tape line.
\item A number of obstacles will be placed on the course. If you exactly follow the tape line you will not run into an obstacle; however, you may have to avoid obstacles as you maneuver away from the tape line to pick up targets.
\end{DoxyItemize}
\item No human will be allowed on the course during a test run.
\item Your robot must be autonomous.
\item All test runs will be video ‘taped.\+’
\item The goal is to maximize your score according to the algorithm discussed below. The maximum score you achieve for any one day over the two days will be your final score.
\item The scores for the entire class will be rank-\/ordered.
\item You will be allowed ten minutes on the course for each test run. This will be strictly timed.
\item Robot
\begin{DoxyItemize}
\item You will be provided with
\begin{DoxyItemize}
\item A basic robot chassis
\item Two motors with encoders and wheels
\item Two motor controllers (H-\/bridges)
\item A robotic arm
\item A battery pack with a power distribution unit
\item Distance sensors.
\item Line sensors
\item Data logger with SD card
\end{DoxyItemize}
\item You do not have to use this robot chassis or arm
\item You will need to supply your own processor(s)
\item You will need to supply your own cameras(s) and cables.
\item You may acquire additional mechanical or electronic parts for your robot.
\item If you plan to spend any money on your robot, you must get permission from me in writing first.
\item Your group has a strict budget of \$300, including any parts that you have already acquired and use on your robot (e.\+g., an Arduino).
\end{DoxyItemize}
\item Rule 8 applies. Rule 8 comes from the official rules for the annual Race to Alaska (see \href{https://r2ak.com/official-rules/}{\tt https\+://r2ak.\+com/official-\/rules/}). Rule 8 states, and I quote\+: If we decide it’s necessary to consult a lawyer to figure out if you are disqualified or not, you are automatically disqualified. Play by the rules and live up to the spirit of the race. If you get cute and push the boundaries, we’ll bring down the hammer. 
\end{DoxyItemize}